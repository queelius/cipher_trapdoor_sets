\documentclass[ ../main.tex]{subfiles}
\providecommand{\mainx}{..}
\begin{document}
\section{Introduction}
The \emph{oblivious set}\cite{oset} is a fundamental data structure that may be used to construct other oblivious object types, like secure indices for Boolean Encrypted Search\cite{sibool}

An \emph{oblivious set} is an oblivious object type for the abstract data type of the \emph{set}. The following operations are defined:
\begin{enumerate}
    \item Approximate membership tests of specific elements may be performed with a false positive rate $\fprate$ and a false negative rate $\fnrate$. Note that there is no way to efficiently iterate over the elements.
    
    \item The cardinality may be estimated to be within some range. The the degree of uncertainty can be made arbitrarily large entropy at the expense of its space complexity.
        
    \item Set-theoretic operations like union, intersection, and complement generate oblivious sets that approximate the true operation as a function of the false positive and false negative rates and the degree of similarity between the exact sets under consideration.
\end{enumerate}

In \cref{sec:ob_set_def}, we precisely define the oblivious set.


In \cref{sec:prob_model}, we derive the probablistic model of \emph{oblivious sets}. In \cref{sec:entropy}, we derive the \emph{entropy} of \emph{oblivious sets}. In \cref{sec:estimators}, we derive estimators of properties of \emph{oblivious sets}. In \cref{sec:shs}, we provide a theoretically optimal implementation of the oblivious set.
\end{document}