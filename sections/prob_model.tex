\documentclass[ ../main.tex]{subfiles}
\providecommand{\mainx}{..}
\begin{document}
\section{Probabilistic model}
\label{sec:prob_model}
The uncertain number of false positives is given by the following theorem.
\begin{theorem}
\label{thm:fpbinom}
Given a set $\Set{S}$ with $m$ positives from a universe of $u$ elements, the number of \emph{false positives} in an approximate set $\ASet{S}$ with a false positive rate $\fprate$ is a random variable denoted by $\FP_m$ with a distribution given by
\begin{equation}
    \FP_m \sim \bindist(u - m, \fprate)\,.
\end{equation}
\end{theorem}

The number of false negatives is given by the following theorem.
\begin{theorem}
\label{thm:fnbinom}
Given a set $\Set{S}$ with $m$ positives, the number of \emph{false negatives} with respect to an approximate set $\ASet{S}$ with a false negative rate $\fnrate$ is a random variable denoted by $\FN_m$ with a distribution given by
\begin{equation}
    \FN_m \sim \bindist(m, \fnrate)\,.
\end{equation}
\end{theorem}

The \emph{expected} cardinality is given by the following theorem.
\begin{theorem}[Cardinality]
Given a set $\Set{S}$ of cardinality $m$ from a universe of $u$ elements, an approximate set $\ASet{S}$ has an \emph{expected} cardinality given by
\begin{equation}
\label{eq:exp_card}
    u \fprate + m (1 - \fprate - \fnrate)\,,
\end{equation}
where $\fprate$ is the false positive rate and $\fnrate$ is the false negative rate.
\end{theorem}

\subsection{Positives and negatives}
The distribution of false positives and false negatives are Bernoulli distributed random variables conditioned on a particular number of positives. The distribution of positives (and negatives) is given by the following definition.
\begin{definition}
The number of \emph{positives} in a universe of $u$ elements is uncertain. We model the uncertainty as a discrete random variable, denoted by $\RV{P}$, with a probability mass function\footnote{The probability mass function of a random variable $\RV{X}$ is denoted by $\PDF{\,\cdot\,}[\RV{X}]$.}
\begin{equation}
    \PDF{p \Given u}[\RV{P}]
\end{equation}
and a support $\{0,\ldots,u\}$. Conversely, the distribution of negatives is a random variable $\RV{N} = u - \RV{P}$ with a probability mass function
\begin{equation}
    \PDF{n \Given u}[\RV{N}] = \PDF{u - n \Given u}[\RV{P}]\,.
\end{equation}
\end{definition}
The form the probability mass function $\PDF{\,\cdot\,}[\RV{P}]$ takes cannot be a priori specified, although it may be estimated with an empirical probability.

Modeling the distribution of positives provides a complete specification for the distribution of false positives and false negatives (and true positives and true negatives).
\begin{example}
The \emph{expected} number of false positives is give by the expectation
\begin{align}
    \Expect{\FP}
        &= \sum_{p=0}^{u} \sum_{f_p=0}^{u - p} f_p \cdot \PDF{p \Given u}[\RV{P}] \PDF{f_p \Given p, u, \fprate}[\FP]\\
        &= \sum_{p=0}^{u} \PDF{p \Given u}[\RV{P}] \Expect{\FP_p \Given u} = \sum_{p=0}^{u} \PDF{p \Given u}[\RV{P}](u-p) \fprate\\
        &= \fprate\left(u - \sum_{p=0}^{u} m \PDF{p \Given u}[\RV{P}]\right) = \fprate\left(u - \Expect{\RV{P}}\right)\,.
\end{align}
Note that $\Expect{\RV{N}} = u - \Expect{\RV{P}}$, thus
\begin{equation}
    \Expect{\FP} = \fprate \Expect{\RV{N}}\,.
\end{equation}
\end{example}

The joint probability mass function of positives, false positives, and false negatives is given by
\begin{equation}
    \PDF{p, f_p, f_n \Given u, \fprate, \fnrate} = \PDF{p \Given u}[\RV{P}] \PDF{f_p \Given p, u, \fprate}[\FP] \PDF{f_n \Given p, u, \fnrate}[\FN]\,.
\end{equation}
\end{document}