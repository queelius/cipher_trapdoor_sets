\documentclass[ ../main.tex]{subfiles}
%\providecommand{\mainx}{..}
\begin{document}
\section{Probabilistic model}
\label{sec:prob_model}
The uncertain number of false positives is given by the following theorem.
\begin{theorem}
\label{thm:fpbinom}
Given a set $\Set{S}$ with $m$ positives from a universe of $u$ elements, the number of \emph{false positives} in an approximate set $\ASet{S}$ with a false positive rate $\fprate$ is a random variable denoted by $\FP_m$ with a distribution given by
\begin{equation}
    \FP_m \sim \bindist(u - m, \fprate)\,.
\end{equation}
\end{theorem}

The number of false negatives is given by the following theorem.
\begin{theorem}
\label{thm:fnbinom}
Given a set $\Set{S}$ with $m$ positives, the number of \emph{false negatives} with respect to an approximate set $\ASet{S}$ with a false negative rate $\fnrate$ is a random variable denoted by $\FN_m$ with a distribution given by
\begin{equation}
    \FN_m \sim \bindist(m, \fnrate)\,.
\end{equation}
\end{theorem}

The joint probability mass function of positives, false positives, and false negatives is given by
\begin{equation}
    \PDF{p, f_p, f_n \Given u, \fprate, \fnrate} = \PDF{p \Given u}[\RV{P}] \PDF{f_p \Given p, u, \fprate}[\FP] \PDF{f_n \Given p, u, \fnrate}[\FN]\,.
\end{equation}
\end{document}