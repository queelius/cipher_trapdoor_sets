\documentclass[ ../main.tex]{subfiles}
\providecommand{\mainx}{..}

\begin{document}
\section{Cipher sets}
\label{sec:ob_set_def}
A set is given by the following definition.
\begin{definition}
A set is an unordered collection of distinct elements from a universe of elements.
\end{definition}

A countable set is a \emph{finite set} or a \emph{countably infinite set}. A \emph{finite set} has a finite number of elements. For example,
\[
    \Set{S} = \{ 1, 3, 5 \}
\]
is a finite set with three elements. A \emph{countably infinite set} can be put in one-to-one correspondence with the set of natural numbers
\begin{equation}
    \NatSet = \{1,2,3,4,5,\ldots\}\,.
\end{equation}

The cardinality of a set $\Set{S}$ is a measure of the number of elements in the set, denoted by
\begin{equation}
    \Card{\Set{S}}\,.
\end{equation}
The cardinality of a \emph{finite set} is a non-negative integer and counts the number of elements in the set, e.g.,
\[
    \Card{\left\{ 1, 3, 5\right\}} = 3\,.
\]


A \emph{cipher} of a value type $T$ is 

Informally, a cipher set of $\Set{S}$, denoted by $\OT{\Set{S}}$, provides a \emph{confidential} in-place binary representation such that very little information about $\Set{S}$ is disclosed.

Note that an oblivious set where elements are from the universe $\Set{U}$ permits membership tests on elements in $\Set{U}$.
These elements, and the universe $\Set{U}$, are \emph{not} necessarily oblivious types.
It is also possible to have an oblivious set over oblivious elements, where the elements have their own set of separate constraints, e.g., an oblivious type $X$ in which only the less-than and equality predicates are possible.
The oblivious set, of course, only allows \emph{membership} tests to be performed on elements over $\OT{\Set{U}}$.

\subsection{Notation}
The binary set $\{0,1\}$ is denoted by $\BitSet$. The set of all bit (binary) strings of length $n$ is therefore $\BitSet^n$. The cardinality of $\BitSet^n$ is
\begin{equation}
\Card{\BitSet^n} = 2^n\,.
\end{equation}
The set of all bit strings of length $n$ or less is denoted by $\BitSet^{\leq n}$, which has a cardinality $2^{n+1}-1$. The countably infinite set of all bit strings, $\BitSet^{\leq \infty}$, is also denoted by $\BitSet^{*}$.

The bit length of an object $x$ is denoted by
\begin{equation}
\BL(x)\,,
\end{equation}
e.g., the bit length of any $x \in \BitSet^n$ is $\BL(x) = n$.

A (convenient) one-to-one correspondence between $\BitSet$ and $\NatSet$ is given by the following definition.
\begin{definition}
	\label{def:mapping}
	Let the set of bit strings $\BitSet^{*}$ and the set of natural numbers 
	$\NatSet$ have the bijection given by
	\begin{equation}
	(b_1 \, b_2 \, \cdots \, b_m) \longleftrightarrow 2^m + \sum_{j=1}^{m} 2^{m - j} b_j\,.
	\end{equation}
	We denote the mapping of a bit string (or natural number) $x$ by $x'$.
\end{definition}
An important observation of this mapping is that a natural number $n$ maps to a bit string $n'$ of length $\BL(n') = \lfloor \log_2 n \rfloor$.

\subsection{Cipher object types}
A \emph{type} is a set and the elements of the set are called the \emph{values} of the type.
An \emph{abstract data type} is a type and a set of operations on values of the type.
For example, the \emph{integer} abstract data type is defined by the set of integers and standard operations like addition and subtraction.
A \emph{data structure} is a particular way of organizing data and may \emph{model} one or more abstract data types.

A \emph{concept} is ...

Suppose we have a conceptual type $T$ where $T$ is some \emph{predicate} such that a concrete object type $x$ models the concept of $T$ if $T(x)$.

If $T$ is a \emph{set} of values then $T(x)$ must be able to construct values of $x$ that model particular values in $T$.
At least conceptually, then, we have an equality predicate (not based on syntax, but semantics) such that if $a$ in $x$ models a particular value $t$ in $T$, we say that $a = t$ is true.

Mathematically, functions may be represented by relations, e.g., $T(x) \implies \Fun{t} \subseteq x \times x$.
However, there are many equivalent \emph{forms}, e.g., the Boolean function $\Fun{and} \colon \BitSet \times \BitSet \mapsto \BitSet$, defined as $\{\Tuple{0,0,0},\Tuple{0,1,0},\Tuple{1,0,0},\Tuple{1,1,1}\}$, has algebraic normal forms $x \land y$ and $(x' \lor y')'$.\footnote{These two expressions are understood with respect to a particular \emph{algebra} known as Boolean algebra.}
We generally define functions with respect to some algebra since they are more \emph{concise} and convey greater meaning.

Suppose we have a well-defined parametric function $\Fun{t} \colon T \mapsto T$ expressed with respect to the computational basis of $T$.
Then, $T(x) \implies \Fun{t} \colon x \mapsto x$.
However, for efficiency, it may be desirable to specialize the definition of $\Fun{t} \colon x \mapsto x$ since $x$ may be more efficiently implemented $\Fun{f}$ using some alternative algorithm.

Cipher object types\cite{obtype} provide guarantees about what can be learned about object types $T$ by observing instance representations, transformations, memory access patterns, or any other possible correlations when the \emph{secret} is not known.

Suppose object types $T$ has the computational basis $\Set{F} \coloneqq \{\Fun{f}[1],\ldots,\Fun{f}[n]\}$ and $\Fun{f}[1]$ is a unary function of type $T \mapsto T$.
Then, an example of a \emph{cipher} of $T$ over $\Fun{f}[1]$ and $\Fun{f}[3]$ is denoted by $(\OT{T},\Fun{\OT{f}}[1])$ where $\Fun{\OT{f}}[1] \colon \OT{T} \mapsto \OT{T}$ and $\Fun{\OT{f}}[j]$ for $j=2,
\ldots,n$ are undefined and, based on its representation, $\forall_{x} \Prob{\OT{T}(x) = x} \leq \frac{1}{\Card{T}}$ and $\Prob{\Fun{\OT{f}}[1] = \Fun{f}[1]} \leq \frac{1}{n}$, unless it is desirable to disclose this information.
Note that, by the property that guessing the value $x$ given $\OT{T}(x)$ is no better than a random guess, observing the representation of $\Fun{\OT{f}}[1]$ must not reveal anything about $T$.
So, for instance, $\OT{T}(x) \overset{\Fun{\OT{f}}[1]}{\mapsto} \OT{T}(y)$ must not reveal anything about $x$ and $y$.
This can be problematic, since, for instance, if $(T,\Fun{f}[1])$ has a unique identity $e$ ($\Fun{f}[1](e) = e$), then $\OT{T}$ must have \emph{multiple} representations of $e$, i.e., $\OT{T}(e)$ is an equivalence class, such that for all $\OT{e} \in \OT{T}(e)$, $\Fun{f}[1](\OT{e}) \neq \OT{e}$.
There are many other examples of this, e.g., algebraic groups may have identities, inverses, and other structure that may reveal something about the values being modeled by the cipher value type.
An alternative approach is to introduce \emph{approximation error} and \emph{noise} such that
\begin{equation}
    \Prob{\Fun{decode}(\OT{x}) \neq x} = \epsilon,
\end{equation}
i.e., the function $\Fun{\OT{f}}[1]$ model a \emph{random approximate map}, e.g., a first-order random approximate map such that $\Prob{\Fun{\OT{f}}(\OT{x}) \neq \Fun{f}(x)} = \epsilon$.
Also, for \emph{noisy} values, that is, cipher values that do not model any elements in $T$, $\Fun{\OT{f}}$ models a random variable over the codomain of $\Fun{\OT{f}}[1]$.

The data constructor
\begin{equation}
    \Fun{ctor} \colon (T + C) \mapsto \NatSet \mapsto \OTL{T+C}
\end{equation}
may also be a random approximate function.

Type information may also be approximate and cipher tags.




Optimally, the only information that can be learned about $\OT{x}$ is given by the well-defined \emph{behavior} of the the abstract data type $T$ on a subset of its computational basis $\Set{F}$.\footnote{
Frequently, some operations may reveal too much information about the values and so are excluded from the computational basis of the oblivious type.}
For instance, say an operator $\Fun{g} \colon T \mapsto \BitSet$ is not in the computational basis of $\OT{T}$, i.e., $\Fun{g} \colon \OT{T} \mapsto \BitSet$ is undefined.
Suppose $\Fun{g}(x) = 1$ for a particular object of $T$ and half of the inputs map to $1$ (and half map to $0$).
If the only information we have about $x$ is given by $\OT{x}$, then $\Prob{\Fun{g}(\OT{x})=1} = 0.5$, i.e., we can do no better than a random guess.

We consider random approximate cipher sets,
\begin{equation}
    (\{\PS{\Set{X}},\Set{X}\},\SetContains)
\end{equation}
where $\SetContains \colon \PS{\Set{X}} \times \Set{X} \mapsto \pnapprox{2}{\cipher{\BitSet}}$.

 of the type $\AT{\OTL{\PS{\OT{\Set{X}}}}}$ where $\Set{X}$ is the underlying universal set of elements, i.e., cipher sets whose bit string representations are uncorrelated with its specific members over cipher elements, frequently denoted \emph{trapdoors}.
This is opposed to, say, $\PS{\OT{\Set{X}}}$ in which we have a regular set over ciphers elements of type $\Set{X}$, or $\OTL{\PS{\Set{X}}}$ cipher sets over elements of type $\Set{X}$.

It is generally easier to understand what we mean by this with respect to a \emph{computational basis}, e.g., the 
\emph{singular hash set} has a computational basis given by
\begin{equation}
	\SetContains \colon \OTL{\PS{\Set{X}}} \times \OT{\Set{X}} \ATOverUnder{\OMap} \BitSet\,.
\end{equation}
Other operations, like set-theoretic operations, may be composed using this computational basis.
Such a composition may contain information about the objective sets, but an explicit composition of cipher sets is still generally considered to be a cipher set.

A more confidential variation of a cipher set is given by
\begin{equation}
	\SetContains \colon \OTL{\PS{\Set{X}}} \times \OT{\Set{X}} \ATOverUnder{\OMap} \OT{\BitSet}\,,
\end{equation}
i.e., its membership predicate returns a Boolean cipher.
Since \emph{representational equality} implies \emph{equality}, there is always, at least implicitly, a predicate for the identity relation of the type
\begin{equation}
	= \colon \OTL{\PS{\Set{X}}} \times \OTL{\PS{\Set{X}}} \mapsto \BitSet\,.
\end{equation}
However, equality does not necessarily imply representational equality, and so there is also an opportunity for a predicate of the type
\begin{equation}
	= \colon \OTL{\PS{\Set{X}}} \times \OTL{\PS{\Set{X}}} \ATOverUnder{\OMap} \OT{\BitSet}\,,
\end{equation}
but this may only be practical for relatively small cipher sets.
If the cipher sets are random approximate sets over a \emph{countably infinite} universe, then \emph{equality} implies \emph{representational equality}.
\end{document}