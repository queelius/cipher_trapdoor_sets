\documentclass[ ../main.tex]{subfiles}

\SetKwFunction{MakeBloomFilter}{make\_bloom_filter}
\SetKwFunction{Encode}{encode}
\SetKwFunction{trunc}{tr}
\SetKwFunction{bf}{bloom\_filter}
\SetKwData{BF}{BF}

\SetKwData{Btm}{0}
\SetKwData{Top}{1}

\providecommand{\mainx}{..}

\begin{document}
\section{Entropy Bloom filters}
\label{sec:bf}
The Bloom filter is a well-known \emph{probabilistic data structure} that models a \emph{second-order} positive Bernoulli set where the probability of a false negative is $0$ and the probability of a false positive may be specified to be some value in $(0,1)$.

We would like to have properties of both homomorphisms $\Fun{F}$ and $\Fun{G}$, but with a much smaller space complexity and much greater entropy (as a proxy of confidentiality).

In what follows, we provide a theoretical implementation of the \emph{cipher set} that obtains optimality in the following ways:
\begin{enumerate}
	\item The entropy obtains the theoretical upper bound.
    \item It is computationally efficient to construct.
    \item It is close to the theoretically optimal lower-bound on space complexity?
\end{enumerate}

\begin{definition}
\label{alg:makeset}
	The \emph{data type} for the Bloom filter is defined as $\BF_{\Fun{F} \Set{X}} \coloneqq \BitSet^k \times \BitSet^*$ with a value constructor $\bf_{\Fun{F}\Set{X}} \colon \PS{\Set{X}} \times \RealSet \mapsto \BF_{\Fun{F} \Set{X}}$ defined as
	\begin{equation}
	\bf_{\Fun{F} \Set{X}}(\Set{A}, \fprate) \coloneqq \Tuple{q,n'}
	\end{equation}
	where
	\begin{equation}
	\begin{split}
	\Set{A}     & = \{x_{(1)},\ldots,x_{(m)}\}\,,\\
	k			& \coloneqq -\log_2 \fprate\,,\\
	\hash_k(b,n)& \coloneqq \trunc\left(\hash\left(b \cat n'\right),k\right)\,,\\	
	\phi(w,n) 	& \coloneqq \hash_k\left(\Fun{E}(w), n\right)\,,\\
	\Omega_l 	& \coloneqq \SetBuilder{\phi(x,l) \in \BitSet^k}{w \in \Fun{F}(\Set{A})}\,,\\
	n 			& \coloneqq \min\SetBuilder{ j \in \NatSet }{ \Omega_j \in \PowerSet(\BitSet^k) \land \Card{\Omega_j} = 1}\,,\\
	q 			& \coloneqq \phi\left(\Fun{F} x_{(1)},n\right)\,.
	\end{split}
	\end{equation}
where $\Fun{F} \colon \Set{X} \mapsto \Set{W}$ is a homomorphism and $\Fun{E} \colon \Set{W} \mapsto \BitSet^*$ is a prefix-free coder.\footnote{If $\Fun{E}$ is not prefix-free, then an approximation error independent of parameter $\fprate$ is introduced.}
\end{definition}
%TODO: randomly sample from a population of bit strings that have the same floor log n.

If $\Fun{F}$ is the \emph{identify} $\Fun{id} \colon \Set{X} \mapsto \Set{X}$ then $\left(\BF_{\Set{X}}, \SetContains\right)$ models the positive, second-order random approximate algebraic structure $\left(\PS{\Set{X}},\PAT{\SetContains}\right)$.
That is, $\bf_{\Set{X}}(\Set{A},\fprate) \sim \PASet{A}[\fprate]$.

However, typically, the singular hash set is intended to be used for as a \emph{cipher set}, a parametric type $\PAT{\OTL{\OT{\Set{X}}}}$ where $\Fun{F} \colon \Set{X} \mapsto \OT{\Set{X}}$, e.g., $x \overset{\Fun{F}}{\mapsto} \OT{x}$ where $\OT{x}$ may be, say, an \emph{equivalence class} $\{\OT{x}_1,\ldots,\OT{x}_k\}$ in which each of the elements map back to $x$ under $\Fun{F}^{-1} \colon \OT{\Set{X}} \mapsto \Set{X}$.

\begin{theorem}
The data type $\BF$ with value constructor $\bf \colon \PowerSet(\BitSet^*) \times \SetBuilder{2^{-k} \in \RatSet}{k \in \NatSet} \mapsto \BF$ over the computational basis $\SetContains \colon \BF \times \BitSet^* \mapsto \BitSet$ is a positive, second-order random approximate set of $\OT{\Set{A}}$ with a \emph{false positive rate} $\fprate$ over $\OT{\Set{X}} \SetDiff \OT{\Set{A}}$.
That is, \emph{a priori},
\begin{equation}
	\bf_{\Set{X}}(\Set{A},\fprate) \sim \PAT{\OT{\Set{A}}}[\fprate]
\end{equation}
for any $\Set{A} \in \PS{\Set{X}}$ and $\fprate \in \SetBuilder{2^{-k} \in \RatSet}{k \in \NatSet}$.
\end{theorem}
\begin{proof}
In order for the value $\bf_{\Set{X}}(\Set{A},\fprate)$ to be a random approximate set with a distribution given by $\PAT{\OT{\Set{A}}}[\fprate]$, it must satisfy two conditions as specified by \cref{def:approx_set}:
\begin{enumerate}
	\item $\OT{\Set{A}}$ is a subset of $\PAT{\OT{\Set{A}}}$.
	This condition guarantees that no \emph{false negatives} may occur.
	\item An element in $\OT{\Set{X}}$ that is not a member of $\OT{\Set{A}}$ is a member 
	of $\PAT{\OT{\Set{A}}}$ with a probability $\fprate$, denoted the false positive rate, 
	i.e.,
	\begin{equation}
	\Prob{x \in \AT{\OT{\Set{A}}} \Given x \notin \OT{\Set{A}}} = \fprate
	\end{equation}
	for any $x \in \Set{X}$.
\end{enumerate}

To prove the first condition, note that \cref{alg:contains} tests any element $x$ for membership in $\PASet{S}$ by computing the hash of $x$ concatenated with the bit string $b_n$ and returning \True if the hash is $h_k$ where \cref{alg:makeset} finds bit strings $b_n$ and $h_k$ such that each element of $\Set{S}$ concatenated with $b_n$ hashes to $h_k$.

To prove the second condition, suppose we have a set $\Set{S} = \{x_1,\ldots,x_m\}$ and each element in $\Set{S}$ hashes to $y = \hash(x_1)$. By \cref{asm:ro_approx}, $\hash \colon \BitSet^* \mapsto \BitSet^k$ approximates a random oracle and thus uniformly distributes over its domain of $2^k$ possibilities. Since $y$ is a particular element in $\BitSet^k$, the probability that an element not in $\Set{S}$ hashes to $y$ is $2^{-k}$.
\end{proof}


Suppose we have a set $\Set{A}$ of $n$ elements, $\Set{A} \coloneqq \{x_{(1)},\ldots,x_{(n)}\}$, from a domain $\Set{X} \coloneqq \{x_1,\ldots,x_u\}$, $u \geq n$, an array of $m$ bits $b_1,\ldots,b_m$, and $k$ random hash functions $\hash_1,\ldots,\hash_k$ of type $\Set{X} \mapsto \{0,\ldots,m-1\}$.
The probability that the $l$-th component of the bit array is $0$ is given by
\begin{equation}
    \gamma_p \coloneqq \Prob{\cap_{j=1}^{n} \cap_{i=1}^{k} \left(\hash_i(x_j) \neq l\right)}\,.
\end{equation}
Assuming the hash functions are random hash functions, a priori each hash function uniformly distributes over $\{0,\ldots,m-1\}$ for each $x \in \Set{X}$.
Therefore, any element $x \in \Set{X}$ does not map to $p$ with probability $p = \frac{m-1}{m}$.
Thus, to have maximum entropy, the bit at position $p$ must be $0$ or $1$ with equal probability,
\begin{equation}
    \Prob{\cap_{j=1}^{n} \cap_{i=1}^{k} p} = \frac{1}{2}\,.
\end{equation}
Since each hash function is independent,
\begin{equation}
    \Prob{\cap_{j=1}^{n} p^k} = \frac{1}{2}\,.
\end{equation}
Since each $\hash(x) \sim \dudist(\{0,\ldots,m-1\})$ for $x \in \Set{X}$,
\begin{equation}
    p^{n k} = \frac{1}{2}\,.
\end{equation}
Substituting $p$ with $(m-1)/m$,
\begin{equation}
    \left(1-\frac{1}{m}\right)^{n k} = \frac{1}{2}\,.
\end{equation}
Solving for $k$ yields the result
\begin{equation}
    k = -\left(n \log_2(1-1/m)\right)^{-1}\,.
\end{equation}
The false positive rate is thus given by
\begin{equation}
    \fprate = \left(1-\exp(-kn/m)\right)^{k}\,.
\end{equation}
Substituting the provided $k$ into the above and simplifying yields
\begin{equation}
    \fprate = \left(1-\exp\!\left( \log_2(1-1/m)\right)^{-1}/m\right)^{-\left(n \log_2(1-1/m)\right)^{-1}}\,.
\end{equation}
If we parameterize $m$ as some factor of $n$, $m \coloneqq \alpha n$, then
\begin{equation}
    \fprate = \left(1-\exp\!\left( \log_2(1-1/\alpha n)\right)^{-1}/\alpha n\right)^{-\left(n \log_2(1-1/\alpha n)\right)^{-1}}\,.
\end{equation}
which, if we take the limit $n \to \infty$, simplifies to
\begin{equation}
    \fprate = 2^{-\alpha \log 2}\,,
\end{equation}
which has a space efficiency of $\alpha$ bits per element.
The Bloom filter obtains an optimal space efficiency of $1 / \log 2$ with an appropriate choice of $k$ and $m$.
If we let $\alpha = 1/\log 2$, we get the result
\begin{equation}
    \fprate = 2^{-1}\,,
\end{equation}
i.e., the entropy Bloom filter provides no better than a random guess over the negative elements.
By comparison, the space-optimal Bloom filter, when given a false positive rate $0.5$, has $\ln 2 \approx 0.7$ bits per element, which is over twice as space efficient.
The information-theoretic lower-bound, given $\fprate=0.5$, is $\log_2 1/2$

If we let $\alpha \to \infty$, then $\fprate \to 0$ but the absolute space efficiency goes to $0$.

In either case, the entropy of the bit string is maximum for its bit length, but since $m = \alpha n$, $n = m / \alpha$.
Therefore, if $\alpha$ is given, we know the cardinality of the set.
If we fix $m$, then we can solve for $k$ to obtain the desired $\fprate$.

We could serialize the Bloom filter for transmission over the wire and reconstruct it on the untrusted machine.
This could be one way of representing \emph{cipher sets}, such as Boolean queries, which permits queries on elements of the cipher set.


More elements may be inserted into the Bloom filter, but it will decrease the entropy and increase the false positive rate.




In particular, we consider a data type that supports both the membership and identity relations while obtaining information-theoretic upper-bound on entropy given a false positive rate or bit length (dependent).

Since it obtains the information theoretic upper-bound, the bit strings generated naturally obtain \emph{maximum entropy} such that each bit is $0$ with probability $1$.
If we use its optimal space complexity (which has an absolute space efficiency of $1.44$), then it is possible to estimate the cardinality of the set with high certainty.\footnote{Of course, the objective set may be poisoned with nonsense values before generating the set such that, for instance, every set is expected to have the same bit length.}


\end{document}  