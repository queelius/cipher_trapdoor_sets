\documentclass[ ../main.tex]{subfiles}

\SetKwFunction{MakeSingularHashSet}{make\_singular\_hash\_set}
\SetKwFunction{Encode}{encode}

\providecommand{\mainx}{..}

\begin{document}
\section{Cipher Boolean ring}
Say we have the Boolean ring
\begin{equation}
	A \coloneqq \left(\PS{\BitSet*}, \triangle, \SetIntersection, \SetComplement, \EmptySet\right)
\end{equation}
where $\BitSet^*$ is the free semigroup over $\BitSet$, which is the closure over concatenation $\cat \colon {\BitSet*} \times {\BitSet*} \mapsto {\BitSet*}$.

Consider the Boolean ring
\begin{equation}
	B \coloneqq \left(\BitSet^m, \oplus, \land, \Fun{id}, 0^m\right)
\end{equation}
and suppose we wish to provide a homomorphism of type $A \mapsto B$ that \emph{only} preserves identities.

\begin{definition}
	The homomorphism $\Fun{G} \colon A \mapsto C$ is defined as
	\begin{equation}
	\Fun{G}(\beta) \coloneqq
	\begin{cases} 
	\quad \hash(x) 		& \qquad \beta \in {\BitSet*}\\
	\quad \oplus 		& \qquad \beta = \triangle \\
	\quad \land 		& \qquad \beta = \SetIntersection\\
	\quad \Fun{id}		& \qquad \beta = \SetComplement\\
	\quad 0^m			& \qquad \beta = \EmptySet\\
	\end{cases}
	\end{equation}
where $\oplus$, $\land$, and $\Fun{id}$ are respectively bitwise \emph{exclusive-or}, bitwise \emph{and}, and \emph{identity}.
\end{definition}

This homomorphism is \emph{one-way} for the same reason as before.
The identity relation
\begin{equation}
	(\Fun{G} \Set{A}) (\Fun{F} \triangle) (\Fun{G} \Set{B}) = \Fun{G}(\Set{A} \triangle \Set{B})
\end{equation}
is guaranteeed by the homomorphism.\footnote{Observe that if $\Set{A}$ and $\Set{B}$ are disjoint, $\Set{A} \triangle \Set{B} = \Set{A} \SetUnion \Set{B}$.}

There is an \emph{isomorphism} between $\Fun{G}$ and $\Fun{F}$ since $\Set{A} \SetUnion \Set{B} \coloneqq \left(\Set{A} \triangle \Set{B}\right) \triangle (\Set{A} \SetIntersection \Set{B})$.
If the trusted system applies only the \emph{group}
\begin{equation}
	D \coloneqq \left(\BitSet^m, \oplus, \Fun{id}, 0^m\right)
\end{equation}
the number of $1$'s and the cardinality of the set has no relation.

Given a set $\{x,y,z\} = \{x\} \SetUnion \{y\} \SetUnion \{z\}$, $\Fun{G}$ maps this to $\hash(x) \oplus \hash(y) \oplus \hash(z)$.

Since the hash function $\hash$ is a random oracle, by the property of the exclusive-or operation each set in $A$ maps to a \emph{random} bit string in ${\BitSet*}$. That is to say, it is indistinguishable from random noise.
However, it only supports \emph{identity} relations.
\begin{theorem}
False positives on the \emph{equality} predicate occurs with probability
\begin{equation}
	\fprate(m) = 2^{-m}\,,
\end{equation}
where $m$ is the bit length of the hash function in the Boolean cipher ring.
\end{theorem}
\begin{proof}
Two sets are said to be identical if they map to the same bit string, and thus, as before, \emph{false positives} are possible,
\begin{equation}
	\Prob{\Fun{G} \Set{X} = \Fun{G} \Set{Y} \Given \Set{X} \neq \Set{Y}}
\end{equation}
which occurs with probability $2^{-m}$ since each set maps to a random bit string of size $m$.
\end{proof}

\end{document}