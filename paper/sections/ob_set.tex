\documentclass[ ../main.tex]{subfiles}
\providecommand{\mainx}{..}

\begin{document}
\section{Cipher sets}
\label{sec:ob_set_def}
A set is given by the following definition.
\begin{definition}
A set is an unordered collection of distinct elements from a universe of elements.
\end{definition}

A countable set is a \emph{finite set} or a \emph{countably infinite set}. A \emph{finite set} has a finite number of elements. For example,
\[
    \Set{A} = \{ 1, 3, 5 \}
\]
is a finite set with three elements. A \emph{countably infinite set} can be put in one-to-one correspondence with the set of natural numbers
\begin{equation}
    \NatSet = \{1,2,3,4,5,\ldots\}\,.
\end{equation}

The cardinality of a set $\Set{B}$ is a measure of the number of elements in the set, denoted by
\begin{equation}
    \Card{\Set{B}}\,.
\end{equation}
The cardinality of a \emph{finite set} is a non-negative integer and counts the number of elements in the set, e.g.,
\[
    \Card{\left\{ 1, 3, 5\right\}} = 3\,.
\]

\subsection{Notation}
The binary set $\{0,1\}$ is denoted by $\BitSet$. The set of all bit (binary) strings of length $n$ is therefore $\BitSet^n$. The cardinality of $\BitSet^n$ is
\begin{equation}
\Card{\BitSet^n} = 2^n\,.
\end{equation}
The set of all bit strings of length $n$ or less is denoted by $\BitSet^{\leq n}$, which has a cardinality $2^{n+1}-1$.

The countably infinite set of all bit strings, $\BitSet^{\leq \infty}$, is also denoted by $\BitSet^*$.
$\BitSet^*$ is the free semigroup over $\BitSet$ which is closed over concatenation $\cat \colon {\BitSet^*} \times {\BitSet^*} \mapsto {\BitSet^*}$.

The bit length of an object $x$ is denoted by
\begin{equation}
\BL(x)\,,
\end{equation}
e.g., the bit length of any $x \in \BitSet^n$ is $\BL(x) = n$.

A (convenient) one-to-one correspondence is given by the following definition.
\begin{definition}
\label{def:mapping}
The free semigroup $\BitSet^{*}$ and the set of natural numbers $\NatSet$ have the bijection $' \colon \BitSet^* \mapsto \NatSet$ and $' \colon \NatSet \mapsto \BitSet^*$ defined as
\begin{equation}
	(b_1 \, b_2 \, \cdots \, b_m) \longleftrightarrow 2^m + \sum_{j=1}^{m} 2^{m - j} b_j\,.
\end{equation}
We denote the mapping of a bit string (or natural number) $x$ by $x'$.
\end{definition}
An important observation of this mapping is that a natural number $n$ maps to a bit string $n'$ of length $\BL(n') = \lfloor \log_2 n \rfloor$.

\subsection{Cipher object types}
A type is an attribute of data that informs how data is intended to be used and constrains the values that an expression, such as function (that may be constructed by a lambda expression), might take.
Common primitive data types include integers, floating-point numbers, characters, and Booleans.
From a mathematical perpsective, a type is a set and the elements of the set are called the \emph{values} of the type.

An \emph{abstract data type} is a type and a set of operations on values of the type.
For example, the \emph{integer} abstract data type is defined by the set of integers and standard operations like addition and subtraction.
A \emph{data structure} is a particular way of organizing data and may \emph{model} one or more abstract data types.

Suppose we have an abstract data type denoted by $T$ with a \emph{computational basis} $\Set{F}$ that are (at least partially) functions of $T$.
We denote that an \emph{object} $x$ in computer memory models $T$ by $T(x)$.

A \emph{cipher object type}\cite{obtype} that models $T$ is a related type denoted by $\OT{T}$ that provides guarantees about what can be learned about $x$ by observing the representation, memory access patterns, or any other possible correlations with respect to $\OT{x}$.

Optimally, the only information that can be learned about $\OT{x} \coloneqq \OT{T(x)}$ is given by the well-defined \emph{behavior} of the the abstract data type $T$ on a subset of its computational basis $\Set{F}$.\footnote{
Frequently, some procedures may reveal too much information about the values and so are excluded or defined in a different way in the computational basis of the cipher type.}
For instance, say an operator $\Fun{g} \colon T \mapsto \BitSet$ is not in the computational basis of $\OT{T}$, i.e., $\Fun{g} \colon \OT{T} \mapsto \BitSet$ is undefined.
Suppose $\Fun{g}(x) = 1$ for a particular object of $T$ and half of the inputs map to $1$ (and the other half map to $0$).
If the only information we have about $x$ is given by $\OT{x}$, then $\Prob{\Fun{g}(\OT{x})=1} = 0.5$, i.e., we can do no better than a random guess.

We consider cipher objects that model the \emph{set-indicator} functions, i.e.,
\begin{equation}
    (\PS{\Set{X}}, \SetContains)\,.
\end{equation}
The set-membership function $\SetContains$ over elements of type $\PS{\Set{X}}$ is sufficient to model the Boolean algebra $(\Set{X},...)$.

Informally, a cipher set of $\Set{C}$, denoted by $\OT{\Set{C}}$, provides a \emph{confidential} in-place binary representation such that very little information about $\Set{C}$ is disclosed.

Note that the cipher type $(\PS{\Set{X}}, \SetContains)$ consists of elements from the universal set $\Set{X}$.
These elements, and the universe $\Set{X}$, are \emph{not} necessarily oblivious types.
If $U$ models values of the universal set $\Set{X}$, then 

It is also possible to have an oblivious set over oblivious elements, where the elements have their own set of separate constraints, e.g., an oblivious type $X$ in which only the less-than and equality predicates are possible.
The oblivious set, of course, only allows \emph{membership} tests to be performed on elements over $\OT{\Set{X}}$.




We consider random approximate cipher sets of the type $\AT{\OTL{\PS{\OT{\Set{X}}}}}$ where $\Set{X}$ is the underlying universal set of elements, i.e., cipher sets whose bit string representations are uncorrelated with its specific members over cipher elements, frequently denoted \emph{trapdoors}.
This is opposed to, say, $\PS{\OT{\Set{X}}}$ in which we have a regular set over ciphers elements of type $\Set{X}$, or $\OTL{\PS{\Set{X}}}$ cipher sets over elements of type $\Set{X}$.

It is generally easier to understand what we mean by this with respect to a \emph{computational basis}, e.g., the 
\emph{singular hash set} has a computational basis given by
\begin{equation}
	\SetContains \colon \OTL{\PS{\Set{X}}} \times \OT{\Set{X}} \ATOverUnder{\OMap} \BitSet\,.
\end{equation}
Other operations, like set-theoretic operations, may be composed using this computational basis.
Such a composition may contain information about the objective sets, but an explicit composition of cipher sets is still generally considered to be a cipher set.

A more confidential variation of a cipher set is given by
\begin{equation}
	\SetContains \colon \OTL{\PS{\Set{X}}} \times \OT{\Set{X}} \ATOverUnder{\OMap} \OT{\BitSet}\,,
\end{equation}
i.e., its membership predicate returns a Boolean cipher.
Since \emph{representational equality} implies \emph{equality}, there is always, at least implicitly, a predicate for the identity relation of the type
\begin{equation}
	= \colon \OTL{\PS{\Set{X}}} \times \OTL{\PS{\Set{X}}} \mapsto \BitSet\,.
\end{equation}
However, equality does not necessarily imply representational equality, and so there is also an opportunity for a predicate of the type
\begin{equation}
	= \colon \OTL{\PS{\Set{X}}} \times \OTL{\PS{\Set{X}}} \ATOverUnder{\OMap} \OT{\BitSet}\,,
\end{equation}
but this may only be practical for relatively small cipher sets.
If the cipher sets are random approximate sets over a \emph{countably infinite} universe, then \emph{equality} implies \emph{representational equality}.
\end{document}